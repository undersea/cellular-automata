% Created 2010-10-13 Wed 14:40
\documentclass[presentation]{beamer}
\usepackage[utf8]{inputenc}
\usepackage[T1]{fontenc}
\usepackage{fixltx2e}
\usepackage{graphicx}
\usepackage{longtable}
\usepackage{float}
\usepackage{wrapfig}
\usepackage{soul}
\usepackage{textcomp}
\usepackage{marvosym}
\usepackage{wasysym}
\usepackage{latexsym}
\usepackage{amssymb}
\usepackage{hyperref}
\tolerance=1000
\providecommand{\alert}[1]{\textbf{#1}}

\title{Using Cellular Automata as a clustering tool}
\author{James Hurford}
\date{13/10/2010}

\usetheme{default}\usecolortheme{default}
\begin{document}

\maketitle




\begin{frame}
\frametitle{Introduction}
\label{sec-1}

\begin{itemize}
\item Data mining (Classification)
\item Purpose is to show that cellular automata can be used as a
   clustering tool
\end{itemize}
\end{frame}
\begin{frame}
\frametitle{Cellular automata}
\label{sec-2}

\begin{itemize}
\item Grid of cells
\item State (0 or 1) determined by simple rules based on immediate neighbours

\begin{itemize}
\item Von Neumann or Moore's neighbourhood
\end{itemize}

\item Leads to complex behaviour
\end{itemize}
\end{frame}
\begin{frame}
\frametitle{Clustering and Classification tool}
\label{sec-3}

\begin{itemize}
\item Emergent behaviour
\item Less error prone
\item Low-bias and Self organising
\end{itemize}
\end{frame}
\begin{frame}
\frametitle{The approach}
\label{sec-4}


\begin{itemize}
\item Uses a majority voting system
\item Grid size and dimensions determined by predicates used
\item It is run until convergence happens

\begin{itemize}
\item This is achieved when all cells are assigned a class
\end{itemize}

\end{itemize}
\end{frame}
\begin{frame}
\frametitle{My implementation}
\label{sec-5}

\begin{itemize}
\item Written in C++ using the STL
\item The process

\begin{itemize}
\item Define a grid

\begin{itemize}
\item Predicates must be turned into integers
\end{itemize}

\item Populate with training data
\item Run until finished
\item Test with test data
\end{itemize}

\end{itemize}
\end{frame}
\begin{frame}[fragile]
\frametitle{Snippet}
\label{sec-6}

\begin{itemize}
\item Neighbour finder
\end{itemize}

\begin{verbatim}
for(unsigned i = 0; i < point.size(); i++) {
  for(int j = -1; j < 2; j+=2) {
    Coord neighbour(point.size());
    neighbour = point;
    neighbour[i] += j;
    if(neighbour[i] >= 0 && neighbour[i] < graph.dimensions[i]) {
      val[graph(neighbour).get()]++;
    }
  }
 }
\end{verbatim}
\end{frame}
\begin{frame}
\frametitle{Findings}
\label{sec-7}

\begin{itemize}
\item In general my data does seem to support Fawcett's hypothesis
\item Does not exactly match up with Fawcett's results
\end{itemize}
\end{frame}
\begin{frame}
\frametitle{Explanation/Evaluation}
\label{sec-8}

\begin{itemize}
\item Inconsistencies between my results and Fawcett's

\begin{itemize}
\item Maybe because of a lack of specific information
\end{itemize}

\item The process can take a long time
\item Memory is the bottleneck
\end{itemize}
\end{frame}
\begin{frame}
\frametitle{Conclusions}
\label{sec-9}

\begin{itemize}
\item Cellular automata can be used as a clustering and classification
   tool
\end{itemize}
\end{frame}
\begin{frame}

  \includegraphics[width=30em \textwidth]{./thankyou.png}
\end{frame}

\end{document}
