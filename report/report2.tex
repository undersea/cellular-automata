% Created 2010-10-13 Wed 10:30
\documentclass[11pt]{article}
\usepackage[utf8]{inputenc}
\usepackage[T1]{fontenc}
\usepackage{fixltx2e}
\usepackage{graphicx}
\usepackage{longtable}
\usepackage{float}
\usepackage{wrapfig}
\usepackage{soul}
\usepackage{textcomp}
\usepackage{marvosym}
\usepackage{wasysym}
\usepackage{latexsym}
\usepackage{amssymb}
\usepackage{hyperref}
\tolerance=1000
\providecommand{\alert}[1]{\textbf{#1}}

\title{Using Cellular Automata for Data Mining}
\author{James Hurford}
\date{}

\begin{document}

\maketitle

\setcounter{tocdepth}{3}
\tableofcontents
\vspace*{1cm}


\section{Introduction}
\label{sec-1}

  A cellular automata is a grid of individual cells whose values are
  updated according to local rules. The rules determining the values
  for each cell are simple, but combined the cells can exhibit
  emergent behaviour. 

  Cellular automata are being used as a modeling tool by
  scientists. The most famous cellular automata is Conway's Game of
  Life, but cellular automata have also been used to model traffic
  flows, chemical reactions, and infections.

  In this report we are using cellular automata for data mining, in
  particular classification, by using cellular automata as a
  clustering tool, based on the article ``Data mining with cellular
  automata'' by Tome Fawcett \cite{fawcett08}.
\section{Background}
\label{sec-2}

  A cellular automata is a grid of individual cells whose values are
  updated according to local rules. The rules determining the values
  for each cell are simple, but combined the cells exhibits emergent
  behaviour.  The cells can be thought of as a population which
  changes at each generation.

  They work with little data and is resistant to errors when noise is
  introduced.  The cells in a cellular automata are self organising
  and have a low-bias when used as a clustering tool.

  Fawcett \cite{fawcett08} was interested in a implementation of
  cellular automata showing it could be used as a data mining,
  classification, clustering tool.  It was not to extend a existing
  method, or to create a new method of data mining. Fawcett's
  \cite{fawcett08} aim was to demonstrate that clustering using
  cellular automata can be done.

  In this model we use the Von Neumann neighbourhood and use a set of
  rules designed to implement a majority vote system.  This has the
  effect of 
\section{Data Mining (Clustering)}
\label{sec-3}

  The process goes like this.  The cellular automata consists of a
  grid of N dimensions, where N is the number of predicates being
  used.  The length of each of the axis used is also determined by the
  size range of the specific predicate used, so a data set with say
  predicates of X and Y, where X has values from zero to ten, and Y
  has values from zero to twenty, so the grid is a two dimensional
  grid of ten by twenty.  Each pair of predicates has a class
  associated with it.  

  If the majority of neighbours values for the
  cell in question have a value of say two, then the value of the
  current cell is changed to two.  If the is a stalemate, such as in a
  two dimensional grid two neighbours having a value of two and two
  other neighbours having a value of one, then the value of the cell
  will be picked at random from the values of the neighbours.
\section{Results}
\label{sec-4}

  
\section{Conclusions}
\label{sec-5}

  


\bibliographystyle{plain}
\bibliography{bibliography}

\end{document}
