% Created 2010-10-11 Mon 17:05
\documentclass[11pt]{article}
\usepackage[utf8]{inputenc}
\usepackage[T1]{fontenc}
\usepackage{fixltx2e}
\usepackage{graphicx}
\usepackage{longtable}
\usepackage{float}
\usepackage{wrapfig}
\usepackage{soul}
\usepackage{textcomp}
\usepackage{marvosym}
\usepackage{wasysym}
\usepackage{latexsym}
\usepackage{amssymb}
\usepackage{hyperref}
\tolerance=1000
\providecommand{\alert}[1]{\textbf{#1}}

\title{Using Cellular Automata for Data Mining}
\author{James Hurford}
\date{}

\begin{document}

\maketitle

\setcounter{tocdepth}{3}
\tableofcontents
\vspace*{1cm}


\section{Introduction}
\label{sec-1}

  Cellular automata (CA) is a discrete, dynamic system, which uses
  localised rules to determine its state.  Individually, at the cell
  level, CA is simple, but collectively demonstrates complex emergent
  behaviour, and is capable of some forms of self organisation.  CA is
  used to model many things, like HIV infections, growth of animal
  habitats.

  Fawcett's \cite{fawcett08} paper uses cellular automata to solve the
  problem of data mining, specifically classification of data using
  cellular automata as a clustering tool.  CA represents a low-bias
  data mining method and that as decisions are made locally, CA has
  virtually no modeling constraints.  CA is a simple, but powerful way
  of gaining fine tuned parallelism on a large scale. So all cells
  can compute their values at the same time.

  The purpose of Fawcett's \cite{fawcett08} article is to demonstrate
  effective generalisation can be achieved as a emergent property of
  cellular automata, from the collective behaviour of simple cells.
  Fawcett \cite{fawcett08} demonstrates that effective classification
  can be achieved similar to that of other data mining models.  This
  is gained from the collective behaviour of simple cells, based on
  local decisions from data gained from the cells immediate
  neighbours.  Fawcett shows that CA perform well with little data and
  is robust in the face of noise.

  We are going to reproduce Fawcett's \cite{fawcett08} work and compare the results,
  demonstrating that his claims are valid.

  
\section{Background}
\label{sec-2}

  Fawcett's \cite{fawcett08} paper describes a proof of concept showing that cellular
  automata can be used for the purpose of data mining.
  
  Cellular automata is a discrete, dynamic system, which uses
  localised rules as a means of determining each cells value.  
  
\section{Data Mining (Clustering)}
\label{sec-3}

  
\section{Results}
\label{sec-4}

  
\section{Conclusions}
\label{sec-5}

  
\section{References}
\label{sec-6}


\bibliographystyle{plain}
\bibliography{bibliography}

\end{document}
