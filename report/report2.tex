% Created 2010-10-14 Thu 15:49
\documentclass[11pt]{article}
\usepackage[utf8]{inputenc}
\usepackage[T1]{fontenc}
\usepackage{fixltx2e}
\usepackage{graphicx}
\usepackage{longtable}
\usepackage{float}
\usepackage{wrapfig}
\usepackage{soul}
\usepackage{textcomp}
\usepackage{marvosym}
\usepackage{wasysym}
\usepackage{latexsym}
\usepackage{amssymb}
\usepackage{hyperref}
\tolerance=1000
\providecommand{\alert}[1]{\textbf{#1}}

\title{Using Cellular Automata for Data Mining}
\author{James Hurford}
\date{}

\begin{document}

\maketitle

\setcounter{tocdepth}{3}
\tableofcontents
\vspace*{1cm}


\section{Introduction}
\label{sec-1}


  Cellular automata are being used as a modeling tool by
  scientists. The most famous cellular automata is Conway's Game of
  Life, but cellular automata have also been used to model traffic
  flows, chemical reactions, and infection rates.

  In this report we are using a cellular automata for data mining, in
  particular classification, by using a cellular automata as a
  clustering tool, based on the article ``Data mining with cellular
  automata'' by Tome Fawcett \cite{fawcett08}.
 
  Fawcett \cite{fawcett08} was interested in a implementation of
  cellular automata showing it could be used as a data mining,
  classification, clustering tool.  It was not to extend a existing
  method, or to create a new method of data mining. Fawcett's
  \cite{fawcett08} aim was to demonstrate that clustering using
  cellular automata can be done.

  The reason Fawcett \cite{fawcett08} is interested in using a
  cellular automata to perform clustering and classification is that a
  cellular automata acts in a similar way to that of nanotechnology. A
  cellular automata can be used as a means of demonstrating how
  nanotechnology might perform data mining.
\section{Background}
\label{sec-2}

  A cellular automata is a grid of cells of finite dimensions.  Each
  cell has a limited number of states.  These states are usually in
  the form of a number like zero for off and one for on.  The grid is
  initialised with each cell being in a initial state such as zero.
  It is then put through a series of steps, called generations and at
  each generation the states of the cells are determined from that
  cells immediate neighbours from a simple set of rules.  A rule could
  be if three or more of my neighbours are in a state of zero then I
  will be one.  There are several forms of neighbourhoods with the two
  most popular being Von Neumann neighbourhood, as shown in figure 1,
  and Moore's neighbourhood as shown in figure 2. At each generation
  the rules applied in parallel, so the states of the cells get
  determined all at the same time.


\begin{figure}[htb]
\centering
\includegraphics[width=10em \textwidth]{von_neumann_ab5c418829cc89af521a5f035cd3478a1d0ab63a.png}
\caption{Von Neumann neighbourhood with neighbours north, south, east and west}
\end{figure}


\begin{figure}[htb]
\centering
\includegraphics[width=10em \textwidth]{moores_8640a89f5d0342f9416f8c47d5c724461bea5c0b.png}
\caption{Moore's neighbourhood with neighbours north, northeast, east, southeast, south, southwest, west and northwest}
\end{figure}
\section{Data Mining (Clustering)}
\label{sec-3}

  The process goes like this.  The cellular automata consists of a
  grid of N dimensions, where N is the number of predicates being
  used.  The length of each of the axis used is also determined by the
  size range of the specific predicate used, so a data set with say
  predicates of X and Y, where X has values from zero to ten, and Y
  has values from zero to twenty, so the grid is a two dimensional
  grid of ten by twenty.  Each pair of predicates has a class
  associated with it.  

  If the majority of neighbours values for the cell in question have a
  value of say two, then the value of the current cell is changed to
  two.  If the is a stalemate, such as in a two dimensional grid two
  neighbours having a value of two and two other neighbours having a
  value of one, then the value of the cell will be picked at random
  from the values of the neighbours. 
 
  They work with little data and is resistant to errors when noise is
  introduced.  The cells in a cellular automata are self organising
  and have a low-bias when used as a clustering tool.

  In this model we use the Von Neumann neighbourhood and use a set of
  rules designed to implement a majority vote system.  This has the
  Manhattan effect on its neighbours
\section{Results}
\label{sec-4}

  
\section{Conclusions}
\label{sec-5}

  
\section{Future work}
\label{sec-6}


\begin{itemize}
\item Extend the cellular automata to find the best predicates and divide
   the data into test and train and use this data to create and test
   itself as a classifier.
\end{itemize}

\bibliographystyle{plain}
\bibliography{bibliography}

\end{document}
